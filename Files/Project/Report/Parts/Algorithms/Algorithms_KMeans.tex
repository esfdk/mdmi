\subsection{K-Means}
\label{Algo_KM}
K-means is a clustering algorithm, which starts out by randomly distributing the elements into clusters. The algorithm then calculates a so-called "centroid" for each cluster. The centroid is calculated from the clustering attributes of each element within that cluster. The algorithm then calculates the distance from each element to all clusters, and moves it into its closest cluster. When all element are moved to their nearest cluster, the centroids are re-calculated. This process is repeated until no elements changes clusters and the state of the clusters is stable.

Once the algorithm is finished, each cluster contains similar elements (according to the clustering attributes). We can use this information to see which elements are similar, and to inject a new element and predict some of its values (values not used for clustering).