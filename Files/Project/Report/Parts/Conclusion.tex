\section{Conclusion}
\label{Con}
\subsubsection*{The Data}
While we believe that the attributes in the data we use present interesting properties, we should probably have included more attributes to make our results more reliable. If we had included more attributes, we would have been able to discretise more than we have without making all the countries look alike. This would have made it easier to find frequent patterns and made our clustering more precise.
\\We could have used attributes that has little (or nothing) to do with economical qualities of a nation. It would still have been possible to frame similar questions to the ones we present in this project if we had done so.
\subsubsection*{Clustering}
Our clustering results may be skewed by the fact that there is a huge variance in the average of unenmployment rate and population numbers. Without normalisation the Euclidean distance between two population sizes becomes much more significant than the distance between two unemployment rates.

When answering the questions posed in the introduction, we are hesitant to rely on the result of algorithms due to the previosuly mentioned unreliability.
\\If we rely on the results of our mining, the countries that never change cluster are Germany, Italy, Turkey and the United Kingdom. As Germany and the United Kingdom both have had very stable economies over the last three decades, it comes as no surprise to us that they stay within the same cluster for the entire period. Italy has had a mostly stable economy over the last thiry years but was hit very hard by the economic crisis of 2007-current\footnote{http://en.wikipedia.org/wiki/Economy\_of\_Italy\#Late\_2000s\_recession}. Turkey is the result that surprises us the most. Turkey, as a nation, has had quite an unstable economy over the past three decades. While benefitting from the Iran-Iraq war of the 1980s, the Turkish economy was battered in 1991 and plunged into crisis in 1994\footnote{http://en.wikipedia.org/wiki/Economic\_history\_of\_Turkey\#Post\_1950}. With such ups and downs for the Turkish economy, it is highly unlikely that Turkey would stay inside the same cluster when clustering mainly on economic attributes\footnote{We believe it is likely that Turkey staying in the same cluster for the entire period is an example of a skewed result caused by population figures.}. 

There were 2(3) nations that often changed clusters in the period: Estonia and Finland (and Malta).
\\Estonia has been one of the fastest growing economies in the in world in modern times\footnote{http://en.wikipedia.org/wiki/Economy\_of\_Estonia\#The\_economy\_today}. The few years it spends in cluster \#3 may be an indicator of the financial crisis.
\\Finland has an outlier in the period 1992-1996, where it belongs to cluster \#1. This may be an indicator of the fact that the Finnish economy fell into a severe recession in 1991 and then joined the European Union in 1995\footnote{http://en.wikipedia.org/wiki/Economy\_of\_Finland\#Liberalization}.
\\Malta has an interesting swing around 2001. Due to the September 11 attacks, the Malta tourist industry suffered a temporary setback - which may be what is reflected here.

It comes as no surprise, to us, that the United States of America is similar to Germany, Italy, Turkey and the United Kingdom considering the possible skewering of results. All three are nations with many inhabitants and relatively stable economies (in 2011).

\subsubsection*{Frequent Patterns}
As we discussed in section \ref{Res_FP}, we found no frequent patterns or association rules with strong support. In the dataset we are working with, a support of exactly 1\% is very weak, and thus the frequent pattern found has very weak support. There we conclude that there are no frequent patterns in the dataset.
\\As we have discussed previously, it is likely that we would be able to find frequent patterns with more support, if we discretised to larger intervals. This would not lead to more reliable results unless more attributes were involved in the mining of a frequent pattern.