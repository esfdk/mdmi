\section{Preprocessing}
\label{PreP}
We use min-max normalisation, discretisation and handling of missing values in this project. We fill in the missing values in the data used for both algorithms. We only do normalisation and discretisation on the data used in the k-means algorithm.

\subsection{Missing values}
When getting our data from Eurostat, missing values were marked with ':'. Since Weka can handle missing values on its own, we just need to convert it, to something Weka can understand ('?'). So as part of our pre-processing, we go through the data, and replace colons with a questionmark.

Depending on the settings, Weka can either ignore the missing values, or use the mean value for that attribute. In our case, we set up Weka to use the mean values. We use the mean values because we rather include data with missing values than to not get results at all (because we limited amounts of entries). 
\\This does bias the data \cite[p. ~90]{DataMining}, which is something to take into consideration when making any conclusions about what the data actually tells us.

\subsection{Normalisation and discretisation}
Our attribute values must be nominal values to be mined for frequent patterns, as it is otherwise highly unlikely that any frequent pattern will be detected. In order to transform the unemployment rate, balance of payments, GDP/inhabitant and population attributes into nominal values, we do the normalisation and discretisation.

We normalise our data within each year from 0 to 5. The maximum value in a specific year will get a new value of 5 - even though another year may have a higher maximum value\footnote{This is to account for inflation.}. This means that even though a maximum value may increase every year, it will still be a 5 if it is the maximum value in other years.

After we have normalised our data, we discretise our data into nominal values by giving them one of 11 values (0.0-0.5, 0.5-1.0, 1.0-1.5, ..., 4.5-5.0, 5.0).